\section{Implementation \& Parameters}

%\subsection{Interlink optimizations}

%% A Merkle tree is used to hash the interlink data structure
%% into a single hash \cite{KLS}. NIPoPoWs form a chain of various levels
%% which can omit blocks. For each block in the proof,
%% only a single pointer needs to be presented to convince the Verifier. Near
%% genesis, the pointers needed correspond to high levels; near
%% the tip, the pointers to low levels. In the construction of
%% the proof, the highest superchain with at least $m$ blocks is included, and
%% assume it is of level $\mu$. The level $\mu - 1$ superchain is fully
%% included and has an expected number of $2m$ blocks. Since all
%% $\mu$-superblocks are also $(\mu - 1)$-superblocks, they only need to be
%% counted once, for $\mu - 1$. Among the expected $2m$ $(\mu - 1)$-superblocks,
%% the last $m$ will be supported by level $\mu - 2$. As before, since $(\mu -
%% 1)$-superblocks are $(\mu - 2)$-superblocks, in expectation only $m$
%% $(\mu-1)$-superblocks are counted. The argument continues inductively, until
%% $2m$ $0$-blocks are included in expectation immediately before the $\chi$
%% suffix. This gives an estimation on the proof size: a total of $m
%% (\log(|\chain|) - \log(m))$ blocks in expectation, $m$ at each of the $\mu - 1$
%% levels and $2m$ $0$-blocks.

We now discuss the size of NIPoPoW proofs and evaluate concrete parameters.
Organizing the interlink data structure as a Merkle tree of $\log(|\chain|)$ items, a
proof-of-inclusion is provided in $\log \log(|\chain|)$ space;
the proof need not include $0$-level pointers, must include the genesis block. The root of the
tree can be proved to be included in the block header in $\log(|\overline
x|)$ using the standard Merkle tree of transactions, where $\overline x$ denotes
the vector of all transactions included in the particular block. This makes the
proof size require $\log(|\overline x|) + \log\log(|\chain|)$ hashes per block
for a total of $m (\log(|\chain|) - \log(m))(\log(|\overline x|) +
\log\log(|\chain|))$ hashes. In addition, $m (\log(|\chain|) - \log(m))$ headers
and coinbase transactions are needed. As an example, given that currently in
bitcoin $|\chain| = 464,185$ and $|\overline x| = 2000$, we have $\log(|\chain|) =
18, \log\log(|\chain|) = 5, \log(|\overline x|) = 11$. For the $k$-suffix, only
$k$ headers are needed. We set $k = 6$ and see that headers are $80$ bytes and
hashes $32$ bytes. For the $k$-suffix as well as the $2m$ $0$-blocks in $\pi$,
neither coinbase data nor prev ids are needed, limiting header size to $48$
bytes. The root and leaves of the pointers tree can be omitted from coinbase
when transmitting the proof. In fact, no block ids need to be transmitted. From
these observations, we estimate our scheme's proof sizes for various
parameterizations of $m$ in Table~\ref{table.size}.

\paragraph{Concrete parameterization.}
To determine concrete values for security parameter $m$, we focus on a
particular adversarial strategy and analyze its probability of success.
%While other adversarial behaviors may be possible, this attack is
%serves as a baseline for determining specific values for
%$m$.
The attack is an extension of the stochastic processes described in
\cite{bitcoin} and \cite{rosenfeld}.

The experiment works as follows: $m$ is fixed and some adversarial computational
power percentage $q$ of the total network computational power is chosen; $k$ is
chosen based on $q$ according to Nakamoto \cite{bitcoin}. The number of blocks
$y$ during which parallel mining will occur is also fixed. The experiment begins
with the adversary and honest parties sharing a common blockchain which ends in
block $B$. After $B$ is mined, the adversary starts mining in secret and in
parallel with the honest parties on her own private fork on top of $B$. She
ignores the honest chain, so that the two chains remain disjoint after $B$. As
soon as $y$ blocks have been mined in total, the adversary attempts a double
spend via a NIPoPoW
by creating two conflicting transactions which are committed to an honest
block and an adversarial block respectively on top of each current chain.
Finally, the adversary mines $k$ blocks on top of the double spending
transaction within her private chain. After these $k$ blocks have been mined,
she publishes her private chain in an attempt to overcome the honest chain.


\begin{table}
  \caption{
    \label{table.size}
    Size of NIPoPoWs applied to Bitcoin today
    ($\approx$450k blocks) for various values of $m$,
    setting $k = 6$.
    %and with current values for transaction count, block count,
    %coinbase size and hash output length.
  }
  \centering
  \begin{tabular}{l|l|l|l}
      {\bf m}  & {\bf NIPoPoW size} & {\bf Blocks} & {\bf
      Hashes}\\
      \hline
      $6$   & $70$  kB & $108$ & $1440$  \\
      $15$  & $146$ kB & $231$ & $2925$  \\
      $30$  & $270$ kB & $426$ & $5400$  \\
      $50$  & $412$ kB & $656$ & $8250$ \\
      $100$ & $750$ kB & $1206$ & $15000$ \\
      $127$ & $952$ kB & $1530$ & $19050$ \\
  \end{tabular}
\end{table}



We measure the probability of success of this attack. We experiment with various
values of $m$ for $y = 100$, indicating $100$ blocks of secret parallel mining.
We make the assumption that honest party communication is perfect and immediate.
We ran $1,000,000$ Monte Carlo executions
\footnote{
\ifanonymous
The link to the source code of our experiment has been redacted for anonymity.
\else
Our experiment can be reproduced by running our code available under
an open source MIT license at
\url{https://github.com/dionyziz/popow/tree/master/experiment}
\fi
}
of the experiment for each value
of $m$ from $1$ to $30$. We ran the simulation for values of computational power
percentage $q = 0.1$, $q = 0.2$ and $q = 0.3$. The results are plotted in
Figure~\ref{fig.nipopow-attack-experiment}.

\begin{figure}
    \caption{\label{fig.nipopow-attack-experiment}
        Simulation results for a private mining attacker with $k$ according to
        Nakamoto and parallel mining parameter $y = 100$. Probabilities in
        logarithmic scale. The horizontal line indicates the threshold
        probability of \cite{bitcoin} is indicated by the horizontal line.
    }
    \centering
    \iftwocolumn
        \includegraphics[width=0.8 \columnwidth,keepaspectratio]{figures/nipopow-attack-experiment.png}
    \else
        \includegraphics[width=0.7\columnwidth,keepaspectratio]{figures/nipopow-attack-experiment-onecolumn.png}
    \fi
\end{figure}

Based on this data, we conclude that $m = 5$ is sufficient to achieve a $0.001$
probability of failure against an adversary with $10\%$ mining power. To secure
against an adversary with more than $30\%$ mining power, a choice of $m = 15$ is
needed.
