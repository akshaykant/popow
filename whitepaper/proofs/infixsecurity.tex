\begin{proof}
Assume a typical execution. To prove the security of the construction, it
suffices to show that if all honest verifiers agree on the value of $Q(\chain)$,
then the verifier will output the same value. Assume that all verifiers agree on
the value $v$.

By Theorem~\ref{thm.security} and because the evaluation of $\tilde\pi$ is
identical in the suffix-sensitive and in the infix-sensitive case, we deduce
that $b = \tilde\pi[-1]$ will be an honestly adopted block. Furthermore, due to
the Common Prefix property of backbone, $b$ will belong to all honest parties'
chains and in the same position, as it is buried under $|\tilde\chi| = k$
blocks.

By assumption, at least one honest party $B$ has provided an honest NIPoPoW
$\pi_B$. Let the chain adopted by that honest party during the round in which
the NIPoPoW was produced be $\chain$. Because the predicate $Q$ is infix-sensitive
and stable, this means that $\exists \chain' \subseteq \chain[:-k]:
P_v(\chain')$. Due to $B$ being honest, $\chain' \subseteq \pi_B$. Let
$S = \textsf{ancestors}(b)$ be the ancestors evaluated by the verifier. Clearly
$\chain' \subseteq S$ and therefore $Q(S) = Q(\chain') = v$.
\Qed
\end{proof}
