\subsection{Succinctness}
\label{sec.succinctness}
Having established security in the general case of the standard honest majority
model, we now concentrate on establishing performance guarantees.

We first observe that full succinctness in the standard honest majority model is
impossible to prove for our construction. To see why, recall that an adversary
with sufficiently large mining power can significantly harm superquality as
described in %Subsection
Section~\ref{subsec.superquality-attack}. By reducing
 superquality for a sufficiently low level $\mu$, for example $\mu = 3$, the
adversary can cause the honest prover to digress in their proofs from high-level
superchains down to low-level superchains, causing a linear proof to be
produced.
For instance, if superquality is harmed at $\mu = 3$, the prover will
need to digress down to level $\mu = 2$ and include the whole $2$-superchain,
which, in expectation, will be of size $|\chain|/2$.

We next establish that in the particular 
``optimistic'' case where the adversary does not use their (minority)
computational power or network power. Therefore, the superquality of the chain
must be the same as a fully honestly-generated chain generated with no network
adversary. Last, for now, we will not allow the adversary to produce any proofs;
that is, all proofs consumed by the verifier are honestly-generated.
We will lift this last assumption shortly.

\begin{restatable}[Optimistic succinctness]{theorem}{restateThmSuccinctness}
    \label{thm.succinctness}
    Non-interactive proofs-of-proof-of-work produced by honest provers in the
    optimistic case are succinct with the number of blocks bounded by $4m
    \log(|\chain|)$, with overwhelming probability in $m$.
\end{restatable}

In the stronger adversarial setting, however, it is possible for the
adversary to produce large dummy (incorrect) proofs that
expand the verification time; security will not be hurt but it would take more time to complete verification. One may dismiss this as a trivial denial of
service attack (and have a resource bounded
verifier simply stop  if it is confronted with such a processing task).
%
Nevertheless, it would be useful for honest provers
to have the ability to signal to the
verifier that such time expansion is indeed necessary because
of an attack on superchain quality, rather than because
a malicious prover is simply sending long proofs that will eventually
be rejected. With such signaling mechanism, a resource bounded verifier can
distinguish between a denial of service attack that may be directed solely to it
from  a denial of service attack that  is launched by an attacker
that has the ability to  interfere globally with superchain quality.

 To facilitate the above signaling, we offer a simple generalization of our
construction that achieves this.
%Our extended construction allows
%the verifier to stop processing
%input early, in a streaming fashion, thereby only requiring logarithmic
%communication complexity per proof received. To achieve
%this, observe that honest proofs need to
%be large only if there is a violation of \textit{goodness}. However, goodness is
%not harmed when the chain is not under attack by the adversarial computational
%power or network. As such,
Specifically, we require the prover to produce a
\textit{certificate of badness} in case there is a violation of
\textit{goodness} in the blockchain.
This certificate is guaranteed to logarithmic, so a verifier can inspect it before processing any more of the proof. If the 
We defer a formal explanation of this mechanism to the full online version.
