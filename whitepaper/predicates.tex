\subsection{Provable chain predicates}

Not all predicates are suitable for our purposes. In this section we characterize the class of
blockchain predicates that will be of interest to us for constructing proofs.
Given that we are in a decentralized setting it can be the case that for certain
sensible blockchain predicate, there will be honest nodes that have not reached
a conclusion about its value. For this reason we allow our predicates to have an
undefined value $\bot$ before they take on a truth value of $0$ or $1$. We
define a number of predicate properties and prove a simple but useful relation
between them.

\begin{definition}{(Monotonicity)}
    A chain predicate $Q(\chain)$ is $\textit{monotonic}$ if for all chains
    $\chain$ and for all blocks $B$ we have that:
    $Q(\chain) \neq \bot \Rightarrow Q(\chain) = Q(\chain B)$.

\noindent
 {(Stability)}
    Parameterized by $k \in \mathbb{N}$, a chain predicate
    $Q$ has $\textit{stability}$ if its value only depends on $\chain[:-k]$.

% \noindent
% {(Persistence)}
%     Parameterized by $k \in \mathbb{N}$, a chain predicate
%     $Q$ has $\textit{persistence}$ if the following is true: When in a certain
%     round the predicate is $k$-stable in some honest party's chain, then
%     whenever it is $k$-stable in any honest party's chain, the truth value of
%     the predicate is the same. Furthermore, the change of truth value from
%     $\bot$ to its  truth value happened at the same blockchain depth.
\end{definition}
%
% \begin{restatable}[Predicate persistence]{theorem}{restateThmPredicatePersistence}
%     If a chain predicate is monotonic, then it satisfies persistence with
%     parameter $k = 2\eta \kappa f$ with overwhelming probability.
% \end{restatable}
% \ifonecolumn
%     \import{./}{proofs/predicatepersistence.tex}
% \else
%     Full proofs are provided in the appendix.
% \fi
