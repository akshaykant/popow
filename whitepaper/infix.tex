\section{Non-interactive blockchain \emph{infix} proofs}
\label{sec:infix}

In the previous section we have seen how to construct proofs for suffix
predicates. As mentioned, the main purpose of this construction is to serve as a
stepping stone for the construction of this section that presents a most useful
class of proofs allow proving more general predicates that can depend on
multiple blocks even buried deep within the blockchain.

More specifically, the generalized prover for \emph{infix proofs} allows
proving any predicate $Q(\chain)$ that depends on a number of blocks that can
appear anywhere within the chain (except the $k$ suffix for stability). These
blocks constitute a \emph{subset} $\chain'$ of blocks, the \emph{witness},
which may not necessarily be a stand-alone subchain. This allows proving
powerful statements such as, for example, whether a transaction is confirmed. We
define next formally the class of predicates that will be of interest.

% XXX extend this class of predicates to include comparison of position
% within the blockchain (these position-dependent predicates may be unprovable
% in velvet mode due to diamond topologies)
\begin{definition}[Infix sensitivity]
\label{def:infix}
A chain predicate $Q_{d,k}$ is \textnormal{infix sensitive} if it can be
written in the form

$$
Q_{d,k}(\chain) =
\begin{cases}
  \text{true, if }
    \exists \chain' \subseteq \chain[:-k]: |\chain'| \leq d \land D(\chain')\\
  \text{false, otherwise}
\end{cases}
$$

Where $D$ is an arbitrary computable predicate.
\end{definition}

Note that $\chain'$ is a blockset and may not necessarily be a blockchain.
Furthermore, observe that for all block sets $\chain' \subseteq \chain$ we have
that $Q(\chain') \Rightarrow Q(\chain)$. This will allow us to later argue that
adding more blocks to a blockchain cannot invalidate its witness.

Similarly to suffix-sensitive predicates, infix-sensitive predicates $Q$ can be
evaluated very efficiently. Intuitively this is possible because of their
localized nature and dependency on the $D(\cdot)$ predicate which requires only
a small number of blocks to conclude whether the predicate should be true.

\noindent\textbf{Example.}
We next show how to express the predicate that asks whether a certain
transaction with id $txid$ has been confirmed as an infix sensitive predicate.
We define the predicate $D^{txid}$ that receives a single block and tests
whether a transaction with id $txid$ is included. The predicate $Q^{txid}_{1,
k}$ is defined as in Definition~\ref{def:infix} using the predicate $D^{txid}$
and the parameter $k$ which in this case determines the desired stability of the
assertion that $txid$ is included (such as, for instance $k = 6$). Note in this
case that auxiliary data will have to be supplied by the prover to aid the
provability of $D$. In particular, for example, the Merkle Tree
proof-of-inclusion path to the Merkle Tree root of transactions will need to be
included in the case of Bitcoin or the Merkle Patricia Trie proof-of-inclusion
path to the Transaction Trie root will need to be included in the case of
Ethereum, similar to an SPV proof. Both of these will be logarithmic in the
number of transactions included in the block and, hence, constant with respect
to the size of the blockchain. In case of a vendor awaiting transaction
confirmation to ship a product, the proof that a certain transaction paid into
a designated address for the particular order should be sufficient. Note that,
in this scheme, it is impossible to determine whether the money has subsequently
been spent by the vendor in a future block, and so can only be used by the
vendor holding the respective secret keys.

In the above example, note that if the verifier outputs \emph{false}, this
behavior will generally be inconclusive in the sense that the verifier could be
outputting \emph{false} either because the payment has not yet been confirmed or
because the payment was never made. We can easily modify the scheme to allow the
payer to claim that the payment was made at some particular block height $\ell$.
The vendor can then bail out after a number of blocks $\ell$ and conclude that
the payment was never made. In order to do that formally, \emph{two} different
infix predicates must be evaluated by the NIPoPoW protocol. The first predicate
$Q_1$ as above simply checks for transaction confirmation. The second predicate
$Q_2$ attests to the size of the underlying blockchain and in particular returns
\emph{true} if the blockchain has grown beyond $\ell$ blocks long. The payment
is deemed successful if $Q_1$ outputs \emph{true} and unsuccessful if $Q_2$
outputs \emph{true}. While both predicates are \emph{false} the result of the
experiment is inconclusive. The predicate $Q_2$ can be implemented in
blockchains which include a verified block depth in their block headers such as
Ethereum. As always, the block whose header is checked for block depth must be
a \emph{stable} block in $\chain[:-k]$ to ensure that a malicious miner is not
able to tamper with it.

\subsection{Construction}

The construction of these proofs is shown in
Algorithm~\ref{alg.nipopow-infix-prover}. The infix prover accepts two
parameters: The chain $\chain$ which is the full blockchain and $\chain'$ which
is a sub-blockset of the blockchain whose blocks are of interest for the
predicate in question. The prover calls the previous suffix prover to produce a
proof as usual. Then, having the prefix $\pi$ and suffix $\chi$ of the suffix
proof in hand, the infix prover adds a few auxiliary blocks to the prefix $\pi$.
The prover ensures that these auxiliary blocks form a chain with the rest of the
proof $\pi$. Such auxiliary blocks are collected as follows: For every block $B$
of the subchain $\chain'$, the immediate previous ($E'$) and next ($E$) blocks
in $\pi$ are found. Then, a chain of blocks $R$ which connects $E$ back to $B'$
is found by the algorithm followDown. If $E'$ is of level $\mu$, there can be no
other $\mu$-superblock between $E'$ and $B'$, otherwise it would have been
included in $\pi$. Therefore, $B'$ already contains a pointer to $E'$ in its
interlink, completing the chain.

\import{./}{algorithms/alg.nipopow-infix-follow.tex}

The way to connect a superblock to a previous lower-level block is implemented
in Algorithm~\ref{alg.nipopow-infix-follow}.  Block $B'$ cannot be of higher or
equal level than $E$, otherwise it would be equal to $E$ and the followDown
algorithm would return. The algorithm proceeds as follows: Starting at block $hi =
E$, it tries to follow a pointer to as far as possible. If following the pointer
surpasses $lo = B'$, then the following is aborted and a lower level is tried,
which will cause a smaller step within the skiplist. If a pointer was followed
without surpassing $B'$, the operation continues from the new block, until
eventually $B'$ will be reached, which concludes the algorithm.

\import{./}{algorithms/alg.nipopow-infix-prover.tex}

An example of the output of followDown is shown in Figure~\ref{fig.infix}. This
is a portion of the proof shown at the point where the superblock levels are at
level $4$. A descend to level $0$ was necessary so that a regular block would be
included in the chain. The level $0$ block can jump immediately back up to level
$4$ because it has a high-level pointer.

\begin{figure}[h]
    \caption{An infix proof descend. Only blue blocks are included in the proof.
    Blue blocks of level $4$ are part of $\pi$, while the blue blocks of level
    $1$ through $3$ are produced by followDown to get to the block of level $0$
    which is part of $\chain'$.}
    \centering
    \iftwocolumn
        \includegraphics[width=\columnwidth,keepaspectratio]{figures/infix.png}
    \else
        \includegraphics[width=0.7\columnwidth,keepaspectratio]{figures/infix.png}
    \fi
    \label{fig.infix}
\end{figure}

The verification algorithm must then be modified as in
\ref{alg.nipopow-verifier-infix}.

\import{./}{algorithms/alg.verifier-infix.tex}

The algorithm works by calling the suffix verifier. It also maintains a blockDAG
collecting blocks from all proofs (it is a DAG because \textit{interlink} can be
adversarially defined). This DAG is maintained in the $\textsf{blockById}$
hashmap. Using it, \textsf{ancestors} uses simple graph search to extract the
set of ancestors of a block present. In the final predicate evaluation, the set
of ancestors of the best blockchain tip is passed to the predicate. The
ancestors are included to avoid an adversary who presents an honest chain but
skips the blocks of interest.
