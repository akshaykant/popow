\section{Blockchain Infix proofs}

\label{sec:infix}

%\subsection{Construction}

In the previous section we have seen how to construct proofs for suffix predicates.
As mentioned, the main purpose of this construction is to serve as a stepping
stone for the construction of this section that presents
a most useful class of
proofs allow proving more general predicates that can depend on multiple blocks
even buried deep within the blockchain.

More specifically, the generalized prover for
\textit{infix proofs} allows proving any predicate $Q(\chain)$ that depends on a
number of blocks that can appear anywhere within the chain (except the $k$
suffix for stability). These blocks constitute a \textit{subset} $\chain'$ of
blocks which may not necessarily be a stand-alone blockchain. This allows
proving powerful statements such as, for example, whether a transaction is
confirmed.
We define next formally the class of predicates that will be of interest.

% XXX extend this class of predicates to include comparison of position
% within the blockchain (these position-dependent predicates may be unprovable
% in velvet mode due to diamond topologies)
\begin{definition}[Infix sensitivity]
\label{def:infix}
A chain predicate $Q_{\ell,d,k}$ is \textnormal{infix sensitive} if it can be
written in the form

$$
Q_{\ell,d,k}(\chain) =
\begin{cases}
  \text{undefined, if } |\chain[:-k]| < \ell \text{, otherwise:}\\
  \text{true, if }
    \exists \chain' \subseteq \chain[:-k]: |\chain'| \leq d \land D(\chain')\\
  \text{false, otherwise}
\end{cases}
$$

Where $D$ is an arbitrary predicate. Note that $\chain'$ is a blockset and may
not necessarily be a blockchain.
\end{definition}

Similarly to suffix-sensitive predicates, infix-sensitive predicates $Q$ can be evaluated very efficiently. Intuitively this is possible because
of their localized nature and dependency on the $D(\cdot)$ predicate which
requires only a small number of blocks to conclude whether the predicate should
be true. Nevertheless, some special care will be needed to ensure the condition
on the length  of the chain with respect to $\ell$ is captured.
