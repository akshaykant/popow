\section{Applications \& Evaluation}
\label{sec.applications}

\subsection{Multi-blockchain wallets}
\label{sec.multichain}
An application of our technique is an efficient multi-cryptocoin client.
Consider a merchant who wishes to accept payments in any cryptocoin, not just
the popular ones. The na\"ive approach would be to install an SPV client for each known cryptocoin.
This approach would entail downloading the header chain for each cryptocoin, and
periodically syncing up by fetching any newly generated block headers. An
alternative would be to use an online service supporting multiple currencies,
but this introduces reliance on a third party (e.g. Jaxx and Coinomi rely on
third party networks).

A NIPoPoW-based client would not download the entire header chain, but would intead only receive NiPoPoW proofs each time a payment is received. When a peer informs the client about a payment, they include a block index $\ell$ and NiPoPoW proof of transaction inclusion. The peer must then query \emph{all} of their connected peers, requesting any better better proof for the same predicate. After waiting a short time period for a response, the client runs the \texttt{verify-infix} routine on all received proofs, and accepts the transaction if the output is \emph{true}. Although initially such proofs must be relative genesis, the client may store the most recently-known ($k$-stable) blockhash for each cryptocoin, such that future payments can include NiPoPoW proofs relative to that. Thus for popluar cryptocoins, the NIPoPoW-based
client downloads nearly every block header, like an ordinary SPV client; but for
cryptocoins used infrequently, the NIPoPoW-based client can skip over most
blocks.

\subsection{Simulation}
We simulated the resources savings resulting from the use of a NIPoPoW-based
client. We model the arrival of payments in each cryptocoin as a Poisson process
and assume that the market cap of a cryptocoin is a proxy for usage. Currently,
a total of 731 cryptocurrencies are listed on coin market
directories\footnote{\url{https://coinmarketcap.com/}}. We narrow our focus to
the 80 cryptocurrencies that have their own PoW blockchains (i.e., no PoS) with
a market cap of over USD \$100,000.

In Table~\ref{tbl.currencies} we show aggregate statistics about these 80
cryptocurrencies, grouped according to the their PoW puzzle. While the entire
chain in Bitcoin only amounts to 40 MB, taken together, the 80 cryptocurrencies
comprise 10 GB of proofs-of-work, and generate 10 MB more each day. In
Table~\ref{tbl.experiment} we show the resulting bandwidth costs from simulating
a period of 60 days with $m=24, k=6$, with varying rates of payments received.
%
For the na\"ive SPV client, the total bandwidth cost is dominated by fetching
the entire chain of headers, which the NIPoPoW client avoids. The marginal
cost for na\"ive SPV depends on the number of blocks generated per day, as well
as the transaction inclusion proofs associated with each payment. The NIPoPoW
based client provides the most savings when the number of transactions per day
is smallest, reducing the necessary bandwidth per day (excluding the initial
sync up) by 90\%.

\begin{table}
  \caption{Cost of header chains for all active PoW-based cryptocoins
           (collected from \url{coinwarz.com})}
  \label{tbl.currencies}
  \small
  \centering
  \begin{tabular}{l|l|l|l}
    {\bf Hash} & {\bf Coins} & {\bf Size today} & {\bf Growth rate}  \\
    \hline
    Scrypt  & 44  & 4.3 GB  & 5.5 MB / day \  \\
    SHA-256  & 15  & 1.4 GB  & 937.0 kB / day \  \\
    X11  & 5  & 581.1 MB  & 556.3 kB / day \  \\
    Quark  & 3  & 647.9 MB  & 518.4 kB / day \  \\
    CryptoNight  & 2  & 199.0 MB  & 115.2 kB / day \  \\
    EtHash  & 2  & 588.6 MB  & 921.6 kB / day \  \\
    Groestl  & 2  & 300.3 MB  & 184.2 kB / day \  \\
    NeoScrypt  & 2  & 310.6 MB  & 153.6 kB / day \  \\
    Others  & 5  & 266.2 MB  & 311.1 kB / day \  \\
    % "Others" above is the sum of these:
    % Equihash  & 2  & 17.7 MB  & 92.2 kB / day \  \\
    % Keccak  & 1  & 161.1 MB  & 115.2 kB / day \  \\
    % X13  & 1  & 30.0 MB  & 57.6 kB / day \  \\
    % Lyra2REv2  & 1  & 57.4 MB  & 46.1 kB / day \  \\
    \hline
    Total  & 80   &  8.5 GB  & 9.2 MB  / day  \\
  \end{tabular}
\end{table}

\begin{table}
  \caption{Simulated bandwidth of multi-blockchain clients after two months (Averaged over 10 trials each)}
  \label{tbl.experiment}
  \small
  \centering
  \begin{tabular}
    {
      l@{\hspace{2pt}}|
      l@{\hspace{2pt}}l@{\hspace{2pt}}|
      l@{\hspace{2pt}}l@{\hspace{2pt}}|
      l@{\hspace{2pt}}}

      \multicolumn{1}{l|}{\bf Daily} & \multicolumn{2}{c|}{\bf Naive SPV} & \multicolumn{2}{c|}{\bf NIPoPoW} \\
      {\bf tx} & {\bf Total} & {\bf (Daily)} & {\bf Total} & {\bf (Daily)} & {\bf Savings} \\
    \hline
    100   &  5.5 GB & (5.5 MB)   & 31.7 MB & (507 kB)   & 99\% (91\%) \\
    500   &  5.5 GB & (5.7 MB)   & 68.2 MB & (1.1 MB)     & 99\% (81\%) \\
    1000  &  5.5 GB & (6.0 MB)   & 99.1 MB & (1.6 MB)     & 98\% (73\%) \\
    3000  &  5.6 GB & (7.0 MB)   & 192 MB& (3.1 MB)     & 97\% (56\%) \\
    \end{tabular}
  \end{table}

\subsection{Certificate Transparency and Catena}
\label{sec.catena}
\anote{Maybe we should cut this section or defer it to appendix.}
Catena~\cite{catena} is an approach for certificate transparency that uses a public proof-of-work blockchain to hold Certificate Authorities (CAs) accountable.
CAs publish commitments to (a Merkle tree over) their SSL certificates in the blockchain. 
Relying parties (e.g., browsers, mobile phones) fetch the proofs-of-work from the blockchain to verify that these commitments have been widely published, preventing equivocation.

Catena clients must currently rely on linear SPV proofs, and would therefore see improved performance immediately if using NIPoPoWs. New clients must currently be boostrapped with the entire 40MB Bitcoin header chain, whereas this can be reduced to $156$ kB when (when $m=15$, see Table~\ref{table.size}).
Catena authors anticipate needing to launch a dedicated
Header Relay Network~\cite{catena} to accommodate the extra bandwidth demands
from new Catena clients.
 Second, the steady state cost of operating a Catena client depends
on how frequently certificate digests are published. For example, one usage
scenario cited by Catena~\cite{catena} is Keybase, a service which publishes
certificate digests every 6 hours. During a 6 hour period, Bitcoin would
generate 6 kilobytes of headers, whereas a NIPoPoW proof covering the same range
would require less than half this size. The savings would increase further if
Catena were implemented using any PoW blockchain with more frequent blocks.

\subsection{Two-way-pegged tokens.}
\label{sec.sidechains}
\anote{Rewrite this entirely based on the cross-chain scenario.}
It is widely known that Bitcoin faces significant scaling
hurdles~\cite{onscaling}, but upgrading Bitcoin is notoriously difficult. A
widely anticipated solution is to treat the Bitcoin blockchain as a host for
``sidechains,'' which are proof-of-work blockchains separate from Bitcoin, but
that can be backed by Bitcoin deposits. Our efficient NIPoPoW construction
solves an open problem posed by Back et al., and thus enables this vision.

\anote{Include a feature comparison, with Rootstock/Federated, Drivechain and BTCRelay}

\begin{table}[t]
  \begin{tabular}{r|rrr}
              & Decentralized & Cost        & \\
    \hline
    Federated\cite{rootstock,federated} 
              &  & $O(1)$ & \\
    Drivechain\cite{drivechain} 
              & x& $O(N)$ & \\
    Ours\cite{nipopow} 
              & x& $O(polylog N)$ & \\
  \end{tabular}
\end{table}

The idea behind sidechains is to implement an SPV verifier for one blockchain
(the ``sidechain'') as a smart contract within another blockchain (the ``host
chain''). An (inefficient) implementation of this idea, called
BTCRelay~\cite{ethereum}, is already running on Ethereum. It comprises a
smart contract that allows users to submit Bitcoin block headers, which it validates and
stores. This allows Ethereum smart contracts to condition their behavior based
on committed Bitcoin transactions. The downside is that every Bitcoin header must be
stored, which is expensive due to the cost of Ethereum blockchain storage (7 cents per kB). If
Bitcoin were upgraded with the interlink data structure, the BTCRelay
functionality could be provided inexpensively.

The most anticipated use of sidechains is to implement a virtual asset on the
sidechain backed by currency on the host chain (a two-way-peg)~\cite{sidechains}. This is accomplished by defining two new transaction
types: ``deposit'' transactions lock coins on the host chain and create new
assets on the side chain; ``withdrawal'' transactions do the opposite. Deposit
transactions committed on the host chain are delivered to the sidechain using a
proof; vice versa for withdrawal transactions. Whereas in the multi-blockchain
wallet application the client queries each of its peers (one of which is assumed
to be honest), in this setting peers must listen for event notifications when a
``withdrawal'' transaction is commited, and if they have a better NiPoPoW proof
that invalidates this claim, they must submit it as a transaction within some
time bound. The smart contract should collect a security deposit from the
withdrawer that can be used to compensate peers for providing invalidation
claims. Succinctness (and certificates of badness) are thus especially important
for sidechains, since the size of the largest proof under normal conditions
should determine what is an acceptable security deposit.
