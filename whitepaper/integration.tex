\section{Applications \& Evaluation}
\label{sec.applications}

\textbf{Multi-blockchain wallets.}
\label{sec.multichain}
One application of our technique is an efficient multi-cryptocurrency client.
Consider a merchant that wishes to accept payments in any cryptocurrency, not
just the popular ones. A na\"ive approach would be for her to  install an SPV
client for each known cryptocurrency. This approach would entail downloading the
header chain for each cryptocurrency, and periodically syncing up by fetching
any newly generated block headers. Another alternative would be to use an online
service supporting multiple currencies, although this introduces reliance on a
third party (existing multi-cryptocurrency apps, such as Jaxx and Coinomi rely
on third party networks).

The NIPoPoW-based client maintains a most recent $k$-stable block hash for each
of its supported cryptocurrencies, initially the genesis block for each. Each
time a payment is received, the client connects to peers on the corresponding
network and asks for a NIPoPoW proof relative to the most recently stored block
hash. For cryptocurrencies where payments are received very frequently, the
NIPoPoW-based client might download nearly every block header, just like an
ordinary SPV client; for cryptocurrencies used infrequently, the
NIPoPoW-based client can skip over most blocks.

\textbf{Simulation.}
We developed a simulation to evaluate the resources savings resulting from the
use of a NIPoPoW-based client. We model the arrival of payments in each
cryptocurrency as a Poisson process and assume that the market cap of a
cryptocurrency is a proxy for usage, i.e. in our model most payments received
are Bitcoin transactions, with transactions from the remaining constituting a
long tail. At the time of writing, a total of 731 cryptocurrencies are listed on
coin market directories\footnote{\url{https://coinmarketcap.com/}}. We narrow
our focus to the 80 cryptocurrencies that have their own PoW blockchains (i.e.,
no PoS) with a market cap of over USD \$100,000.

In Table~\ref{tbl.currencies} we show aggregate statistics about these 80
cryptocurrencies, grouped according to the their PoW puzzle. While the entire
chain in Bitcoin only amounts to 40 MB, taken together, the 80 cryptocurrencies
comprise 10 GB of proofs-of-work, and generate 10 MB more each day. In
Table~\ref{tbl.experiment} we show the resulting bandwidth costs from simulating
a period of 60 days with $m=24, k=6$, with varying rates of payments received.

For the na\"ive SPV client, the total bandwidth cost is dominated by fetching
the entire chain of headers, which the NIPoPoW client does not do. The marginal
cost for na\"ive SPV depends on the number of blocks generated each day, as well
as the transaction inclusion proofs associated with each payment. The NIPoPoW
based client provides the most savings when the number of transactions per day
is smallest, reducing the necessary bandwidth per day (not including the initial
sync up) by 90\%.

\begin{table}
  \caption{Cost of header chains for all active PoW-based cryptocoins
           (collected from \url{coinwarz.com})}
  \label{tbl.currencies}
  \small
  \centering
  \begin{tabular}{l|l|l|l}
    {\bf Hash} & {\bf Coins} & {\bf Size today} & {\bf Growth rate}  \\
    \hline
    Scrypt  & 44  & 4.3 GB  & 5.5 MB / day \  \\
    SHA-256  & 15  & 1.4 GB  & 937.0 kB / day \  \\
    X11  & 5  & 581.1 MB  & 556.3 kB / day \  \\
    Quark  & 3  & 647.9 MB  & 518.4 kB / day \  \\
    CryptoNight  & 2  & 199.0 MB  & 115.2 kB / day \  \\
    EtHash  & 2  & 588.6 MB  & 921.6 kB / day \  \\
    Groestl  & 2  & 300.3 MB  & 184.2 kB / day \  \\
    NeoScrypt  & 2  & 310.6 MB  & 153.6 kB / day \  \\
    Others  & 5  & 266.2 MB  & 311.1 kB / day \  \\
    % "Others" above is the sum of these:
    % Equihash  & 2  & 17.7 MB  & 92.2 kB / day \  \\
    % Keccak  & 1  & 161.1 MB  & 115.2 kB / day \  \\
    % X13  & 1  & 30.0 MB  & 57.6 kB / day \  \\
    % Lyra2REv2  & 1  & 57.4 MB  & 46.1 kB / day \  \\
    \hline
    Total  & 80   &  8.5 GB  & 9.2 MB  / day  \\
  \end{tabular}
\end{table}

\begin{table}
  \caption{Simulated bandwidth of multi-blockchain clients after two months (Averaged over 10 trials each)}
  \label{tbl.experiment}
  \small
  \centering
  \begin{tabular}
    {
      l@{\hspace{1pt}}|
      l@{\hspace{1pt}}l@{\hspace{1pt}}|
      l@{\hspace{1pt}}l@{\hspace{1pt}}|
      l@{\hspace{0.1pt}}}

      \multicolumn{1}{l|}{\bf Daily} & \multicolumn{2}{c|}{\bf Naive SPV} & \multicolumn{2}{c|}{\bf NIPoPoW} \\
      {\bf tx} & {\bf Total} & {\bf (Daily)} & {\bf Total} & {\bf (Daily)} & {\bf Savings} \\
    \hline
    100   &  5.5 GB & (5.5 MB)   & 31.7 MB & (507 kB)   & 99\% (91\%) \\
    500   &  5.5 GB & (5.7 MB)   & 68.2 MB & (1.1 MB)     & 99\% (81\%) \\
    1000  &  5.5 GB & (6.0 MB)   & 99.1 MB & (1.6 MB)     & 98\% (73\%) \\
    3000  &  5.6 GB & (7.0 MB)   & 192 MB& (3.1 MB)     & 97\% (56\%) \\
    \end{tabular}
  \end{table}

\textbf{Certificate Transparency. }
\label{sec.catena}
In Catena~\cite{catena} the Bitcoin blockchain is used as an
equivocation-resistant public log in which to publish SSL certificate
commitments. The client is based on the BitcoinJ library, and therefore requires
downloading the entire chain. Catena could therefore immediately be improved
using NIPoPoWs. The Catena authors anticipate needing to launch a dedicated
Header Relay Network~\cite{catena} to accommodate the extra bandwidth demands
from new Catena clients. A variant based on NIPoPoWs could obviate this, since
it eliminates the need to bootstrap new clients by transmitting the entire 40MB
header chain. Second, the steady state cost of operating a Catena client depends
on how frequently certificate digests are published. For example, one usage
scenario cited by Catena~\cite{catena} is Keybase, a service which publishes
certificate digests every 6 hours. During a 6 hour period, Bitcoin would
generate 6 kilobytes of headers, whereas a NIPoPoW proof covering the same range
would require less than half this size. The savings would increase further if
Catena were implemented using any PoW blockchain with more frequent blocks.

\textbf{Sidechains.}
\label{sec.sidechains}
It is widely known that Bitcoin faces significant scaling
hurdles~\cite{onscaling}, but upgrading Bitcoin is notoriously difficult. A
widely anticipated solution is to treat the Bitcoin blockchain as a host for
``sidechains,'' which are proof-of-work blockchains separate from Bitcoin, but
that can be backed by Bitcoin deposits. Our efficient NIPoPoW construction
solves an open problem posed by Back et al., and thus enables this vision.

The idea behind sidechains is to implement an SPV verifier for one blockchain
(the ``sidechain'') as a smart contract within another blockchain (the ``host
chain''). An (inefficient) implementation of this idea, called
BTCRelay~\cite{ethereum}, is already running on Ethereum. It is an Ethereum
smart contract allowing users to submit Bitcoin PoW, which it validates and
stores. This allows Ethereum smart contracts to condition their behavior based
on Bitcoin transactions. The downside is that every Bitcoin header must be
stored. Currently the cost of Ethereum blockchain storage 7 cents per kB. If
Bitcoin were upgraded with the interlink data structure, the BTCRelay
functionality could be provided cheaply.

The anticipated use of sidechains is to implement a virtual asset on the
sidechain backed by currency on the host chain (a two-way
peg)~\cite{sidechains}. This is accomplished by defining two new transaction
types: ``deposit'' transactions lock coins on the host chain and create new
assets on the side chain; ``withdrawal'' transactions do the opposite. Deposit
transactions committed on the host chain are delivered to the sidechain using an
SPV proof; vice versa for withdrawal transactions. To confirm transaction
inclusion, NiPoPoW can be used with an infix predicate --- the predicate claims
that a transaction occurs at a particular height. Lest the host chain receives a
fraudulent NiPoPoW withdrawal, the host chain must wait for a period of time
during which honest parties may submit a better proof. Succinctness is important
for sidechains, since the proofs must be included in the host chain and are thus
costly.
