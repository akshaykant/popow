\subsection{Security}

\begin{restatable}{theorem}{restateThmSecurity}
    \label{thm.security}
    The non-interactive proofs-of-proof-of-work construction for $k$-stable
    monotonic suffix-sensitive predicates is secure with overwhelming
    probability in $\kappa$.
\end{restatable}

We will first give a draft of the proof before getting into the technical
details. Suppose an adversary produces a proof $\pi_\mathcal{A}$ and an honest
party produces a proof $\pi_B$ such that the two proofs cause the predicate $Q$
to evaluate to different values, while at the same time all honest parties have
agreed that the correct value is the one obtained by $\pi_B$. Because of
bitcoin's security, $\mathcal{A}$ will be unable to make these claims for an
actual underlying 0-level chain. We now argue that the operator $\leq_m$ will
signal in favour of the honest parties.

Suppose $b$ is the LCA block between $\pi_\mathcal{A}$ and $\pi_B$. If the chain
forks at $b$, there can be no more adversarial blocks after $b$ than honest
blocks after $b$, provided there are at least $k$ honest blocks (due to the
Common Prefix property). We will now argue that, further, there can be no more
disjoint $\mu_\mathcal{A}$-level superblocks than honest $\mu_B$-level
superblocks after $b$.

To see this, let $b$ be an honest block generated at some round $r_1$ and let
the honest proof have been generated at some round $r_3$. Then take the sequence
of consecutive rounds $S = (r_1, \cdots, r_3)$. Because the verifier requires at
least $m$ blocks from each of the provers, the adversary must have $m$
$\mu_\mathcal{A}$-superblocks in $\pi_\mathcal{A}\{b:\}$ which are not in
$\pi_B\{b:\}$. Therefore, using a negative binomial tail bound argument, we see
that $|S|$ must be long; intuitively, it takes a long time to produce a lot of
blocks $|\pi_\mathcal{A}\{b:\}|$. Given that $|S|$ is long and that the honest
parties have more mining power, they must have been able to produce a longer
$\pi_B\{b:\}$ argument (of course, this comparison will have the superchain
lengths weighted by $2^{\mu_\mathcal{A}}, 2^{\mu_B}$ respectively). To prove
this, we use a binomial tail bound argument; intuitively, given a long time
$|S|$, a lot of $\mu_B$-superblocks $|\pi_B\{b:\}|$ will have been honestly
produced.

We therefore have a fixed value for the length of the adversarial argument, a
negative binomial random variable for the number of rounds, and a binomial
random variable for the length of the honest argument. By taking the
expectations of the above random variables and applying a Chernoff bound, we see
that the actual values will be close to their means with overwhelming
probability, completing the proof.

The main issue motivating the design of PoPoW is to prevent Bahack-style
attacks~\cite{bahack}, where the adversary constructs ``lucky'' high-difficulty
superblocks without filling in the underlying proof-of-work in the lower
levels.
%
%Observe that, while setting $m = 1$ ``preserves'' the proof-of-work in
%the sense that expectations remain the same, the probability of an adversarial
%attack becomes approximately proportional to the adversary power if the
%adversary follows a suitable strategy (for a description of such a strategy,
%see the parameterization section). With higher values of $m$, the probability of
%an adversarial attack drops exponentially in $m$, even though they maintain constant
%computational power, and hence satisfy a strong notion of security.
%
%
Our protocol retains this highlevel approach, but adds a defense against the double-spending attack of Section~\ref{sec:attack}. The attack is neutralized since our
verifier takes ``goodness'' into account when comparing forks.
%blockchains into account when choosing levels, 
%and the verifier may compare proofs at the right level.
