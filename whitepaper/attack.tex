\section{An attack against the PoPoW of \cite{KLS}}

In this section, we revisit the construction for interactive proofs of
proof-of-work from \cite{KLS} and its security.  Their construction is the
starting point for our non-interactive proofs of proof-of-work. We show that the
construction is susceptible to  a double-spending attack even in the case the
adversary controls a minority of the hashing power.

We first show that a powerful attacker can break chain superquality with
non-negligible probability and we construct a concrete attack based on this
observation. Maintaining chain superquality is not in the original
security model. However, we show how the property affects the security of the
underlying proofs. We do this by presenting a concrete attack against their
scheme that allows an adversary to double spend with non-negligible probability.
This attack works under the assumption that the adversary has sufficient hashing
power, which is still below $50\%$.

\noindent
\textbf{The interactive proofs of proof-of-work protocol.}
The interactive proof of proof-of-work of \cite{KLS} works as follows. The
verifier distinguishes between two candidate proofs $(\pi_A, \chi_A)$ and
$(\pi_B, \chi_B)$. If $\pi_A = \pi_B$, the decision can be drawn immediately
(without interaction) based on $\chi_A,\chi_B$. Otherwise, the verifier queries
the two provers for their claimed anchored superchains $\pi_A\upchain^\mu$ and
$\pi_B\upchain^\mu$ at a certain level $\mu$. The verifier starts querying at
the highest possible level $\mu$ and successively descends until level $\mu$ is
sufficiently low such that block $b = LCA(\pi_A\upchain^\mu, \pi_B\upchain^\mu)$
is $m$ blocks deep for at least one of the provers. More specifically, the
querying stops at such $\mu$ when $max(|\pi_A\upchain^\mu\{b:\}|,
|\pi_B\upchain^\mu\{b:\}|) \geq m$. Subsequently, the winner is designated as
the prover with the most blocks after $b$ at that level. More precisely, the
winner is $A$ if $|\pi_A\upchain^\mu\{b:\}| \geq |\pi_B\upchain^\mu\{b:\}|)$ and
otherwise it is $B$. The communication overhead is reduced by only requesting
blocks after the purported LCA.

\noindent
\textbf{Attacking chain superquality.}
\label{subsec.superquality-attack}
Let us now construct an adversary to break this property at level $\mu$. Suppose
the adversary $\mathcal{A}$ controls a portion $t/n$ of the hashing power.
$\mathcal{A}$ works as follows. Assume she wants to attack the honest party $B$
in order to have him adopt a chain $\chain_B$ which has a bad distribution of
superblocks, i.e. such that local goodness is violated in some sufficiently long
subchain. She continuously determines the current chain $\chain_B$ adopted by
$B$. The adversary starts playing after $|\chain_B| \geq 2$. If
$\textit{level}(\chain_B[-1]) < \mu$, then $\mathcal{A}$ remains idle. However,
if $\textit{level}(\chain_B[-1]) \geq \mu$, then $\mathcal{A}$ attempts to mine
an adversarial block $b$ on top of $\chain_B[-2]$. If she is successful, then
she attempts to mine another block $b'$ on top of $b$. If she is successful, she
broadcasts both $b, b'$. The adversarial mining continues until party $B$ adopts
a new chain, which can be due to two reasons: Either the adversary managed to
successfully mine $b, b'$ on top of $\chain_B[-2]$ and succeeded in having $B$
adopt it; or one of the honest parties was able to mine a block which was
subsequently adopted by $B$. In either case, the adversary continues with the
strategy by inspecting $\chain[-1]$ and acting accordingly.

Assume now that an honestly-generated $\mu$-superblock has been adopted by $B$
at position $\chain_B[i]$ at some round $r$. Let us now examine the probability
that $\chain_B[i]$ will remain a $\mu$-superblock in the long run. Suppose $r' >
r$ is the first round after $r$ during which any block is generated. It is clear
that $\mathcal{A}$ will succeed in this attack with non-negligible probability
and cause $B$ to abandon the $\mu$-superblock from their adopted chain.
Therefore, there will be some $\delta$ such that the adversary will be able to
cause such variance with non-negligible probability in $m$. This suffices to
show that superquality is harmed by this attack.

\noindent
\textbf{A double-spending attack. }
Extending the above attack, we modify the superquality attacker into an
attacker that can cause a double spending attack in the proof of
proof-of-work construction. As before, $\mathcal{A}$ targets the proofs
generated by the honest party $B$ by violating $\mu$-superquality in $B$'s
adopted chain. $\mathcal{A}$ begins by remaining idle until a certain chosen
block $b$. After block $b$ is produced, $\mathcal{A}$ starts mining a secret
chain which forks off from $b$ akin to a selfish mining attacker~\cite{selfish}.
The adversary performs a normal spending transaction on the honestly adopted
blockchain and has it confirmed in the block immediately following block $b$.
She also produces a double spending transaction which she secretly confirms in
her secret chain in the block immediately following $b$.

$\mathcal{A}$ keeps extending their own secret chain as usual. However, whenever
a $\mu$-superblock is adopted by $B$, she temporarily pauses mining in
her secret chain and devotes her mining power to harm the $\mu$-superquality
of $B$'s adopted chain. Intuitively, for large enough $\mu$, the time spent
trying to harm superquality will be limited, because the probability of a
$\mu$-superblock occurring will be small. Therefore, the adversary's superchain
quality will be larger than the honest parties' superchain quality (which will
be harmed by the adversary) and therefore, even though the adversary's
0-level blockchain will be shorter than the honest parties' 0-level blockchain,
the adversary's $\mu$-superchain will be longer than the honest parties'
$\mu$-superchain.

The formal calculation of the probability of this attack succeeding is in the
Appendix. However, we note here that, for appropriate choice of system
parameters, the attack can be made to succeed with overwhelming probability.
