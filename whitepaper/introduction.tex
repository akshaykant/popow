\section{Introduction}

Cryptocurrencies such as Bitcoin~\cite{bitcoin}\cite{bitcoinsoftware} and
Ethereum~\cite{ethereum} are peer-to-peer networks that maintain a globally
consistent transaction ledger, using a consensus protocol based on proof-of-work
(PoW) puzzles~\cite{pow,hashcash}. Worker nodes called ``miners'' expend
computational work in order to reach agreement on the state of the network.
Clients on the network, such as mobile phone apps, must verify these
PoW in order to determine the correct view of the network's state, as necessary
to transmit and receive payments.

In this work we introduce, analyze and instantiate a new primitive,
Non-Interactive Proofs of Proof-of-Work (NIPoPoWs), which can be adapted into
existing cryptocurrencies to support more efficient clients. A traditional
blockchain client must verify the entire chain of proofs-of-work, which grows
linearly over time. On the contrary, clients based on NIPoPoWs require resources
only logarithmic in the length of the blockchain.

\noindent
\textbf{Motivation.}
There are three trends that motivate the need for more efficient clients,
specifically clients that can verify transactions without having to process the
whole blockchain. First, while Bitcoin remains by far the largest and most
widely used cryptocurrency, the ecosystem has become significantly more diverse.
Today there are hundreds of competitors, commonly called ``altcoins'', and
Bitcoin's dominance in the market is at an all time low
~\cite{marketcap}.\footnote{Using
market capitalization as a measure of economic size is debatable but it still indicates the direction of the evolving landscape of cryptocurrencies towards multiple offerings.} We envision that our protocol will
form the basis of an efficient multi-blockchain client, which could efficiently
support payments using hundreds of separate cryptocurrencies. This requires
optimizing the ``SPV client'' described in the original Bitcoin
paper~\cite{bitcoin} which, while quite efficient, requires processing an amount
of data growing linearly with the size of the chain $|\chain|$.

Second, cryptocurrencies are increasingly used as components of larger systems
(the cryptocurrency system is typically called a ``blockchain'' when its
currency feature is not salient). As one example, a recent system called
Catena~\cite{catena} uses the bitcoin blockchain as an authenticated log; the
browser embeds a Bitcoin client that uses the blockchain to validate HTTPS
certificates. The bandwidth costs that are tolerable for a dedicated Bitcoin
wallet are likely unacceptable in such embedded contexts. Our techniques can
directly reduce the cost of Catena and similar systems.

Finally, there has been significant interest in the development of
``cross-chain'' applications, i.e. logical transactions that span multiple
separate blockchains. Simple cross-chain transactions are feasible today: the
most well-known is the atomic exchange ~\cite{tiernolan} e.g. a trade of
Bitcoins for Ether. However, more sophisticated applications could be enabled by
our protocol, which would allow the blockchain of one cryptocurrency to embed a
client of a separate cryptocurrency. This concept, initially popularized by a
proposal by Back et al. ~\cite{sidechains}, can be used to avoid a difficult
upgrade process: a new blockchain with additional features, such as experimental
opcodes, can be backed by deposits in the original Bitcoin currency, obviating
the need to bootstrap new currencies from scratch. The ability to use sidechains
as an upgrade and experimentation mechanism has been heralded as a key enabler
of scalability in the Bitcoin ecosystem; however, the sidechains proposal is
incomplete, as it does not provide a protocol that satisfies a natural security
definition, i.e. where attacks succeed with only negligible probability, even
though several attempts were made~\cite{pos,compactspv}. Our work is the
first to answer the open problem posed by Back et al. Sidechains can also be
used as a mechanism to off-load some of the workload incurred  by the bitcoin
network by moving coins to dedicated sidechains, which can later be moved back.

These examples illustrate that our solution is a key component for two important
pillars needed for next-generation blockchains: \textit{interoperability} and
\textit{scalability}.

\noindent
\textbf{Our contributions.}
In summary, we make the following contributions in this paper.
Our main technical contribution is the introduction and instantiation
of a new cryptographic
primitive called Non-Interactive Proofs of Proof of Work (NIPoPoW).
%
We present a formal model and a provably secure instantiation of NIPoPoWs in the
backbone model \cite{backbone}. Our contribution builds on previous work  of \cite{KLS} who posed {\em interactive} proofs of proof-of-work, which, in turn,
are based on previous discussion of such concepts in the bitcoin forums
\cite{highway}. We in fact show an explicit attack against the construction of \cite{KLS} that showcases  the difficulty of designing such protocols. It follows
that our construction is the first secure proof-of-proof-of-work.
Furthermore, our solution has the additional property of being non-interactive,
as the previous construction could be forced into interaction by an adversarial
prover. Similar to previous work, we prove that the proofs are
optimistically succinct meaning that, in honest conditions, the proofs are
logarithmic in size. Improving previous work, we show that optimistic
succinctness can be achieved for adversarially-generated proofs for a number
of blockchain predicates that are high value use cases. Our definition
fills the gap in terms of security modelling and design that existed in previous
proposals, especially the notion of cumulative ``Dynamic Member Multisignature''
~\cite{sidechains}.

We provide concrete parameterization and empirical analysis focusing on showing
the potential savings of our approach versus existing clients. Using actual data
from the bitcoin blockchain, we quantify the actual savings of NIPoPoWs over the
previous known techniques of constructing efficient SPV verifiers.

We describe two practical deployment paths for our NIPoPoWs that existing
cryptocurrencies can adopt to support our efficient clients. Specifically, we
first outline how NiPoPoW's can be deployed using either a ``soft fork'' or a
``hard fork'' upgrade procedure, both of which have been successfully used by
existing cryptocurrencies~\cite{sok}. We then present a less disruptive update
mechanism that we term a ``velvet fork'' and which may be of independent
interest. In a velvet fork update the clients that have ``forked'' from the
original implementation continue to be fully compatible with un-updated clients.
It follows that, in a velvet fork, the blockchain system can remain supported by
a diverse software codebase indefinitely, while it can still enjoy, at least in
proportion, some of the (efficiency in our case) benefits of the update without
any of the security downsides.
