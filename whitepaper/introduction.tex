\section{Introduction}


%% must verify the entire chain of proofs-of-work, which grows
%% linearly over time. On the contrary, clients based on NIPoPoWs require resources
%% only logarithmic in the length of the blockchain.

%
Today, Bitcoin and Ethereum remain the two largest proof-of-work
cryptocurrencies (by market cap). However, the ecosystem has grown diverse, with
dozens of viable ``altcoin'' competitors.
%
Given such an environment,  it becomes increasingly  important to be able to
efficiently handle  multiple blockchains by the same client and reliably
transfer assets between them.
%
%
%
%We envision our NIPoPoW protocol will form the basis of an
%efficient multi-blockchain client, which could efficiently support payments
%using hundreds of different cryptocurrencies.
%Our protocol improves the performance of client.
%blockchain protocols in general
%are increasingly used as components of larger systems.

The first objective requires optimizing the ``SPV client'' described in the
original Bitcoin paper~\cite{bitcoin} which requires processing an amount of
data growing linearly with the size of the blockchain.
%
%% Cryptocurrencies such as Bitcoin~\cite{bitcoin}\cite{bitcoinsoftware} and
%% Ethereum~\cite{ethereum} are peer-to-peer networks that maintain a globally
%% consistent transaction ledger, using a consensus protocol based on proof-of-work
%% (PoW) puzzles~\cite{pow,hashcash}. Worker nodes called ``miners'' expend
%% computational work in order to reach agreement on the state of the network.
%% Clients on the network, such as mobile phone apps, must verify these
%% PoWs in order to determine the correct view of the network's state, something necessary
%% to transmit and receive payments correctly.

%% In this work we introduce, analyze and instantiate a new primitive,
%% Non-Interactive Proofs of Proof-of-Work (NIPoPoWs), which can be adapted into
%% existing cryptocurrencies to support more efficient clients. A traditional
%% blockchain client in order to check a certain blockchain property \anote{TODO: too abstract
%
The second objective, has received  significant attention in the context of
``cross-chain'' applications, i.e. logical transactions that span multiple
separate blockchains. Simple cross-chain transactions are feasible today: the
most well-known is the atomic exchange~\cite{tiernolan}, e.g., a trade of
bitcoin for ether. However, more sophisticated applications could be enabled by
a more efficient proof process, which would allow the blockchain of one
cryptocurrency to embed a client of a separate cryptocurrency. This concept,
initially popularized by a proposal by Back et al. ~\cite{sidechains} can be
used to avoid a difficult upgrade process: a new blockchain with additional
features, such as experimental opcodes, can be backed by deposits in the
original bitcoin currency, obviating the need to transfer capital to the new
cryptocurrency. As one example of cross-chain interfacing, we describe an
initial coin offering (ICO) which distributes tokens issued on one blockchain,
but allows paying for them using coins in another blockchain.

% These examples illustrate that our solution is a key component for two important
% pillars needed for next-generation blockchains: \textit{interoperability} and
% \textit{scalability}. While we use bitcoin concretely as an example, any
% proof-of-work cryptocurrency can adopt our techniques.

\subsection{Our contributions}
Our main technical contribution is the introduction and instantiation of a new
cryptographic primitive called \textit{Non-Interactive} Proofs of Proof-of-Work
(NIPoPoW).

We present a formal model and a provably secure instantiation of NIPoPoWs. Our contribution builds on previous work  of 
of
backbone model \cite{backbone} in terms of modeling and~\cite{KLS} who introduced the concept of (interactive) Proofs of Proof-of-Work,
which, in turn, are based on previous discussion of such concepts in the bitcoin
forums \cite{highway}. In fact, we present an attack against the
construction of \cite{KLS} that can be mounted by an adversary with sufficient, but still less than $50\%$, of hashing power. 
As a result  our construction is the first Proof
of Proof-of-Work (regardless of interactivity) that is secure assuming honest majority.
Furthermore, our solution is non-interactive making it the first protocol
of this kind. 

Regarding to the predicates that are to be demonstrated, 
 previous work allowed only proving that the $k$-sized \emph{suffix} of the
currently adopted blockchain is as claimed. We generalize this notion to prove
any \emph{predicate} across a class of predicates which we call \emph{infix
sensitive}. This enables proving powerful statements pertaining to the
blockchain such as the fact that a transaction took place, that a smart contract
method ran with certain parameters, or that a payment was made into an account.
The most basic application of such proofs, payment verification, require more
general predicates than what is covered in previous work, and we enable these.

Similar to previous work, we prove that the proofs are optimistically
succinct meaning that, in honest conditions, the proofs are logarithmic in size.
Improving previous work, we show that, in the optimistic model of no adversarial
mining power, succinctness can be achieved for even
\textit{adversarially-generated} proofs by introducing the novel concept of
\textit{certificates of badness}. Our definition fills the gap in terms of
security modeling and design that existed in previous proposals, e.g., the
notion of cumulative ``Dynamic Member Multisignature'' ~\cite{sidechains}.

We provide concrete parameterization and empirical analysis focusing on showing
the potential savings of our approach versus existing clients. Using real data
from the Bitcoin network and other blockchains, we quantify the actual savings of
NIPoPoWs over the previous known techniques of constructing efficient SPV
verifiers.

In summary, we make the following contributions:
\begin{enumerate}
  \item We construct the first \emph{provably secure} Proofs of Proof-of-Work.
  \item We make them \emph{non-interactive}.
  \item We describe an \emph{attack} against the previously known proof-of-proof-of-work construction.
  \item We extend proofs to prove \emph{generic infix predicates} pertaining to
        transactions deep within the blockchain.
  \item We improve \emph{succinctness} of previous proofs by weakening the
        optimality assumptions.
  \item We provide \emph{experimental data} which measure the efficiency and
        security of our scheme as well as concrete parameters based on these
        experiments.
\end{enumerate}
