\section{Introduction}


%% must verify the entire chain of proofs-of-work, which grows
%% linearly over time. On the contrary, clients based on NIPoPoWs require resources
%% only logarithmic in the length of the blockchain.

%
Today, Bitcoin and Ethereum remain the two largest
proof-of-work cryptocurrencies (by market cap). However, the ecosystem has grown diverse,
with dozens of viable ``altcoin'' competitors.
%
Given such an environment,  it becomes increasingly  important to
be able to efficiently handle  multiple blockchains by the same client
and reliably transfer assets between them. 
%
%
%
%We envision our NiPoPoW protocol will form the basis of an
%efficient multi-blockchain client, which could efficiently support payments
%using hundreds of different cryptocurrencies.
%Our protocol improves the performance of client.
%blockchain protocols in general
%are increasingly used as components of larger systems.

The first objective requires optimizing the ``SPV
client'' described in the original Bitcoin paper~\cite{bitcoin} which requires
processing an amount of data growing linearly with the size of the blockchain.
%
%% Cryptocurrencies such as Bitcoin~\cite{bitcoin}\cite{bitcoinsoftware} and
%% Ethereum~\cite{ethereum} are peer-to-peer networks that maintain a globally
%% consistent transaction ledger, using a consensus protocol based on proof-of-work
%% (PoW) puzzles~\cite{pow,hashcash}. Worker nodes called ``miners'' expend
%% computational work in order to reach agreement on the state of the network.
%% Clients on the network, such as mobile phone apps, must verify these
%% PoWs in order to determine the correct view of the network's state, something necessary
%% to transmit and receive payments correctly.

%% In this work we introduce, analyze and instantiate a new primitive,
%% Non-Interactive Proofs of Proof-of-Work (NIPoPoWs), which can be adapted into
%% existing cryptocurrencies to support more efficient clients. A traditional
%% blockchain client in order to check a certain blockchain property \anote{TODO: too abstract
%
The second objective, has received  significant attention in the context of 
``cross-chain'' applications, i.e. logical transactions that span multiple
separate blockchains. Simple cross-chain transactions are feasible today: the
most well-known is the atomic exchange~\cite{tiernolan}, e.g., a trade of
bitcoin for ether. However, more sophisticated applications could be enabled 
by a more efficient proof process, which would allow the blockchain of one cryptocurrency to embed a
client of a separate cryptocurrency. This concept, initially popularized by a
proposal by Back et al. ~\cite{sidechains} 
can be used to avoid a difficult
upgrade process: a new blockchain with additional features, such as experimental
opcodes, can be backed by deposits in the original bitcoin currency, obviating
the need to transfer capital to the new cryptocurrency. As one example of
cross-chain interfacing, we describe an initial coin offering (ICO) which distributes tokens issued on
one blockchain, but allows paying for them using coins in another blockchain.

% These examples illustrate that our solution is a key component for two important
% pillars needed for next-generation blockchains: \textit{interoperability} and
% \textit{scalability}. While we use bitcoin concretely as an example, any
% proof-of-work cryptocurrency can adopt our techniques.

\subsection{Our contributions}
In summary, we make the following contributions.
Our main technical contribution is the introduction and instantiation
of a new cryptographic
primitive called Non-Interactive Proofs of Proof-of-Work (NIPoPoW).
%
We present a formal model and a provably secure instantiation of NIPoPoWs in the
backbone model \cite{backbone}. Our contribution builds on previous work  of \cite{KLS} who introduced
the concept of (interactive) proofs of proof-of-work, which, in turn,
are based on previous discussion of such concepts in the bitcoin forums
\cite{highway}. We in fact show an explicit attack against the construction of \cite{KLS} that showcases the difficulty of designing such protocols. It follows
that our construction is the first secure Proof of Proof-of-Work assuming honest majority.
Furthermore, our solution has the additional property of being non-interactive,
as the previous construction could revert into interaction by an adversarial
prover. Similar to previous work, we prove that the proofs are
optimistically succinct meaning that, in honest conditions, the proofs are
logarithmic in size. Improving previous work, we show that optimistic
succinctness can be achieved for adversarially-generated proofs for a number
of blockchain predicates that are high value use cases. Our definition
fills the gap in terms of security modelling and design that existed in previous
proposals, e.g., the notion of cumulative ``Dynamic Member Multisignature''
~\cite{sidechains}.

We provide concrete parameterization and empirical analysis focusing on showing
the potential savings of our approach versus existing clients. Using real data
from the Bitcoin network and other blockchains, we quantify the actual savings of
NIPoPoWs over the previous known techniques of constructing efficient SPV
verifiers.

We describe two practical deployment paths for our NIPoPoWs that existing
cryptocurrencies can adopt: First using either a ``soft fork'' or a ``hard
fork'' upgrade procedure, both of which have been successfully used by existing
cryptocurrencies~\cite{sok}. Second, using a less disruptive update mechanism
that we term a ``velvet fork'' and which may be of independent interest. In a
velvet fork update, the clients that have ``forked'' from the original
implementation continue to be fully compatible with un-updated clients. It
follows that, in a velvet fork, the blockchain system can remain supported by a
diverse software codebase indefinitely, while it can still enjoy, at least in
proportion, some of the (efficiency in our case) benefits of the update without
any of the security downsides.
