\section{Cross-chain ICO evaluation}
In order to support the cross-chain ICO application and to present concrete data
about the cost of executing NIPoPoW-style proofs, we implemented the NIPoPoW
verifier algorithm as a Solidity smart contract
\footnote{
The link to the source code of the smart contract has been redacted for anonymity.
}
The contract consists of two functions.
The \texttt{submit\_nipopow} function is used by the provers to provide their
proof vectors. Instead of passing the block headers of the proof, the provers
pass the hashes of the block headers and the hashes of the interlink vector. The
reason is that the full data of the block header (especially the Merkle tree
root) is only useful for the blocks of interest. Thus, we reduce the amount of
data needed for the proof by a factor of 2. The rest of the parameters are used
in the inclusion proof of the block. After confirming the validity of the proof,
the \texttt{compare\_proofs} function is called between the current and the best
proof. If the current proof is better then it is assigned to the best proof in
the contract's storage.
%The last step is to update the ancestors of the best
%proof.
%Regardless of whether the proof is better it may contain blocks that are
%in the honest chain and are blocks of interest.
The gas costs are summarized in
Table~\ref{tbl:gascosts}. The \$USD column represents the current price of this
much gas on Ethereum.
%The Solidity code listing will be made available in the full online version.

\begin{table}[ht]
  \centering
  \caption{Verifier contract functions}
  \label{tbl:gascosts}
  \begin{tabular}{l|l|l|ll}
    \hline
    Function & Data & Gas cost & \$USD \\ \hline
    \texttt{compare\_proofs} & $\sim$8Kb & $\sim$5M & \$4 \\ \hline
    \texttt{submit\_nipopow} & $\sim$65Kb & $\sim$40M & \$32 \\ \hline
  \end{tabular}
\end{table}
