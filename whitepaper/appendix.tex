% \section{Backbone integration}
% \label{sec.appendix-backbone}
%
% %We use this model to explore which predicates are reliable and can be proven
% %succinctly.
%
% In this section, we illustrate the detailed formalisms to integrate our proof
% systems with the backbone model \cite{backbone}.
%
% The backbone protocol of a miner node is shown in Algorithm~\ref{alg.backbone}.
% The honest miner maintains the longest chain from the network and tries to mine
% on top of it. In Algorithm~\ref{alg.backbone-prover}, we illustrate our new
% entity for the model, the full node or \textit{prover}.  While the full node is
% not mining, it maintains a state with the longest chain from the network.
% Furthermore, whenever it is asked to prove a predicate $Q$ about its local
% chain $\chain$, it calls a Prove function to provide the proof.  We leave this
% Prove function undefined here, as it is part of the concrete protocol
% construction. On the other hand, in Algorithm~\ref{alg.generic-verifier} we
% illustrate another new entity to the model, the generic \textit{verifier},
% which is stateless, but receives proofs from the provers on the network (via
% the environment) and takes a decision about a predicate on the blockchain it
% believes to be the longest.
%
% \import{./}{algorithms/alg.backbone.tex}
% \import{./}{algorithms/alg.backbone-prover.tex}
% \import{./}{algorithms/alg.verifier-framework.tex}
%
% The prover which additionally maintains the \textit{blockById},
% \textit{depth} and \textit{realLink} data structures is illustrated in
% Algorithm~\ref{alg.backbone-velvet-prover}.
%
% \import{./}{algorithms/alg.backbone-velvet-prover.tex}

\noindent
\textbf{Attack probabilities.}
% \textbf{Remark.} It is worth isolating the mistake in the security proof from
% the interactive construction paper \cite{KLS}. Suppose player $B$ is honest and
% player $\mathcal{A}$ is adversarial and suppose $b$, the LCA block, was honestly
% generated and suppose that the superchain comparison happens at level $\mu$.
% Their proof then argues that there will have been more honestly- than
% adversarially-generated $\mu$-superblocks after block $b$. Nevertheless, we
% observe that the mere fact that there have been more honestly- than
% adversarially-generated $\mu$-superblocks after $b$ does not imply that
% $|\overline\pi_\mathcal{A}[\mu]\{b:\}| \leq |\overline\pi_B[\mu]\{b:\}|$. The
% reason is that some of these superblocks could belong to blocktree forks that
% have been abandoned by $B$. Thus, the security conclusion does not follow.
% Regardless, their basic argument and construction is what we will use as a basis
% for constructing a system that is both provably secure and succinct under the
% same assumptions, albeit requiring a more complicated proof structure to obtain
% security.
We calculate the attack probabilities precisely.
% \footnote{We wish to thank Giorgos Panagiotakos, Peter Ga\u{z}i and Nikos
% Leonardos for their insights and feedback here.}.
We simplify the attack to ease the probabilistic analysis. The attack is
parameterized by  parameters $r, \mu$ which are picked by the adversary. $\mu$
is the superblock level at which the adversary will produce a proof longer than
the honest proof. The modified attack works as follows: Without loss of
generality, fix block $b$ to be Genesis. The adversary always mines on the
secret chain which forks off from genesis, unless a \textit{superblock
generation event} occurs. If a superblock generation event occurs, then the
adversary pauses mining on the secret chain and attempts a \textit{block
suppression attack} on the honest chain. The adversary devotes exactly $r$
rounds to this suppression attack; then resumes mining on the secret chain. We
show that, despite this simplification (of fixing $r$) which is harmful to the
adversary, the probability of a successful attack is non-negligible for certain
values of the protocol parameters.

The adversary monitors the network for superblock generation events. Whenever an
honest party diffuses an honestly-generated $\mu$-superblock at the end of a
given round $r_1$, the superblock generation event will have occurred and the
adversary will starting devoting their mining power to block suppression
starting from the next round.

A block suppression attack works as follows. Let $B$ be the honestly generated
$\mu$-superblock which was diffused at the end of the previous round. If the
round was not uniquely successful, let $B$ be any of the diffused
honestly-generated $\mu$-superblocks. Let $B$ be the tip of an honest chain
$\chain_B$. The adversary first mines on top of $\chain_B[-2]$. If she is
successful in mining a block $B'$, she continues extending the chain ending
at $B'$ (to mine $B''$ and so on). The value $r$ is fixed, so the adversary
devotes exactly $r$ rounds to this whole process; the adversary will keep mining
on top of $\chain_B[-2]$ (or one of the adversarially-generated extensions of
it) for exactly $r$ rounds, regardless of whether $B'$ or $B''$ have been found.
At the same time, the honest parties will be mining on top of $B$ (or a
competing block in the case of a non-uniquely successful round). Again, further
successful block diffusion by the honest parties shall not affect that the
adversary is going to spend exactly $r$ rounds for suppression.
This attack will succeed with overwhelming probability for the right choice
of protocol values.

\begin{theorem}[Double-spending attack]
There exist parameters $p, n, t, q,  \mu, \delta$, with $t\leq (1-\delta)(n-t)$,
and a double spending attack against KLS PoPoW that succeeds with overwhelming
probability.
\end{theorem}
\import{./}{proofs/attack.tex}

\noindent
\textbf{Interlink optimizations.}
We now discuss the manner in which the interlink data structure is hashed for
coinbase inclusion. A Merkle tree is used to hash the interlink data structure
into a single hash \cite{KLS}. NIPoPoWs form a chain of various levels
which can omit blocks. For each block in the proof,
only a single pointer needs to be presented to convince the Verifier. Near
genesis, the pointers needed correspond to high levels; near
the tip, the pointers to low levels. In the construction of
the proof, the highest superchain with at least $m$ blocks is included, and
assume it is of level $\mu$. The level $\mu - 1$ superchain is fully
included and has an expected number of $2m$ blocks. Since all
$\mu$-superblocks are also $(\mu - 1)$-superblocks, they only need to be
counted once, for $\mu - 1$. Among the expected $2m$ $(\mu - 1)$-superblocks,
the last $m$ will be supported by level $\mu - 2$. As before, since $(\mu -
1)$-superblocks are $(\mu - 2)$-superblocks, in expectation only $m$
$(\mu-1)$-superblocks are counted. The argument continues inductively, until
$2m$ $0$-blocks are included in expectation immediately before the $\chi$
suffix. This gives an estimation on the proof size: a total of $m
(\log(|\chain|) - \log(m))$ blocks in expectation, $m$ at each of the $\mu - 1$
levels and $2m$ $0$-blocks.

Organizing interlink into a Merkle tree of $\log(|\chain|)$ items, a
proof-of-inclusion is provided in $\log \log(|\chain|)$ space; $0$-level
pointers need not be included in it, but the genesis block does. The root of the
tree can be proved to have been included in the block header in $\log(|\overline
x|)$ using the standard Merkle tree of transactions. This makes the proof size
require $\log(|\overline x|) + \log\log(|\chain|)$ hashes per block for a total
of $m (\log(|\chain|) - \log(m))(\log(|\overline x|) + \log\log(|\chain|))$
hashes. In addition, $m (\log(|\chain|) - \log(m))$ headers and coinbase
transactions are needed. Concretely, given that currently in bitcoin $|\chain| =
464,185$ and $|\overline x| = 2000$, we have $\log(|\chain|) = 18,
\log\log(|\chain|) = 5, \log(|\overline x|) = 11$. For the $k$-suffix, only $k$
headers are needed. We set $k = 6$ and see that headers are $80$ bytes and
hashes $32$ bytes. For the $k$-suffix as well as the $2m$ $0$-blocks in $\pi$,
neither coinbase data nor prev ids are needed, limiting header size to $48$
bytes. The root and leaves of the pointers tree can be omitted from coinbase when
transmitting the proof. In fact, no block ids need to be transmitted. From these
observations, we estimate our scheme's proof sizes for various parameterizations
of $m$ in Table~\ref{table.size}.

\noindent
\textbf{Proof of security. }
Assume $t$ adversarial and $n$ total parties, each with $q$
PoW random oracle queries per round. We will call a query to the
RO $\mu$-\textit{successful} if the RO returns a value $h$
such that $h \leq 2^{-\mu}T$.

We define boolean random variables $X_r^\mu$, $Y_r^\mu$ and $Z_r^\mu$. Fix some
round $r$, query index $j$ and adversarial party index $k$ (out of $t$). If at
round $i$ an honest party obtains a PoW with $id < 2^{-\mu}T$, set $X_r^\mu =
1$, otherwise $X_r^\mu = 0$. If at round $r$ exactly one honest party obtains a
PoW with $id < 2^{-\mu}T$, set $Y_r^\mu = 1$, otherwise $Y_r^\mu = 0$. If at
round $r$ the $j$-th query of the $k$-th corrupted party is $\mu$-successful,
set $Z^\mu_{ijk} = 1$, otherwise $Z^\mu_{ijk} = 0$. Let $Z^\mu_r = \sum_{k=1}^t
\sum_{j=1}^q Z^\mu_{ijk}$. For a set of rounds $S$, let $X^\mu(S) = \sum_{r \in
S} X_r$ and similarly define $Y^\mu(S), Z^\mu(S)$.

\begin{definition} %{(Typical execution)}
    An execution of the protocol is $(\epsilon, \eta)$-\textit{typical} if:

    \textnormal{\bf Block counts don't deviate.}
    For all $\mu \geq 0$ and any set $S$ of consecutive rounds
    with $|S| \geq 2^\mu \eta\kappa$, we have:

    \begin{itemize}
        \item $(1 - \epsilon)E[X^\mu(S)] < X^\mu(S) < (1 + \epsilon)E[X^\mu(S)]$ and $(1 - \epsilon)E[Y^\mu(S)] < Y^\mu(s)$.
        \item $Z^\mu(S) < (1 + \epsilon)E[Z^\mu(S)]$.
    \end{itemize}

    \textnormal{\bf Round count doesn't deviate.}
    Let $S$ be a set of consecutive rounds such that $Z^\mu(S)
    \geq k$ for some security parameter $k$. Then $|S| \geq (1 -
    \epsilon)2^\mu\frac{k}{pqt}$ with overwhelming probability in $k$.

    \textnormal{\bf Chain regularity.}
    No insertions, no copies, and no predictions \cite{backbone} have occurred.
\end{definition}

\begin{theorem}[Typicality]
\label{thm.typicality}
Executions are $(\epsilon, \eta)$-typical with overwhelming
probability in $\kappa$.
\end{theorem}
\import{./}{proofs/typicality.tex}

The following lemma is at the heart of the security proof that will follow.

\begin{lemma}\label{lem.level-comparison}
    Suppose $S$ is a set of consecutive rounds $r_1 \ldots r_2$
    and $\chain_B$ is
    a chain adopted by an honest party at round $r_2$ of a typical execution.
    Let $\chain_B^S = \{b \in
    \chain_B: b \textnormal{ was generated during } S\}$. Let $\mu_\mathcal{A},
    \mu_B \in \mathbb{N}$. Suppose $\chain_B^S\upchain^{\mu_B}$ is good.
    Suppose $\chain'_\mathcal{A}$ is a $\mu_\mathcal{A}$-superchain containing
    only adversarially generated blocks generated during $S$ and suppose that
    $|\chain'_\mathcal{A}| \geq k$.
    Then
    $2^{\mu_\mathcal{A}}|\chain'_\mathcal{A}| < \frac{1}{3}2^{\mu_B}|\chain_B^S\upchain^{\mu_B}|$.
\end{lemma}
\import{./}{proofs/levelcomparison.tex}

\begin{definition}[Adequate level of honest proof]
Let $\pi$ be an honestly generated proof constructed upon some adopted chain
$\chain$ and let $b \in \pi$.

Then $\mu'$ is defined as  $\mu' = \max\{\mu: |\pi\{b:\}\upchain^\mu| \geq
\max(m + 1, (1 - \delta)2^{-\mu}|\pi\{b:\}\upchain^\mu\downchain|)\}$.
We call $\mu'$ the \textit{adequate} level of proof $\pi$ with respect to
block $b$ with security parameters $\delta$ and $m$. Note that the adequate
level of a proof is a function of both the proof $\pi$ and the chosen block $b$.
\end{definition}

\begin{restatable}{lemma}{restateThmAllBlocks}\label{lem.allblocks}
Let $\pi$ be some honest proof generated with security parameters $\delta, m$.
Let $\chain$ be the underlying chain, $b \in \chain$ be any block and $\mu'$ be
the adequate level of the proof with respect to $b$ and the same security
parameters.

Then $\chain\{b:\}\upchain^{\mu'} = \pi\{b:\}\upchain^{\mu'}$.
\end{restatable}
\import{./}{proofs/allblocks.tex}

\begin{lemma}
Suppose the verifier evaluates $\pi_\mathcal{A} \geq \pi_B$ in a protocol
interaction where $B$ is honest and assume during the comparison that the
compared level of the honest party is $\mu_B$. Let $b =
\textsf{LCA}(\pi_\mathcal{A}, \pi_B)$ and let $\mu_B'$ be the adequate
level of $\pi_B$ with respect to $b$. Then $\mu_B' \geq \mu_B$.
\end{lemma}
\import{./}{proofs/leveladequacy.tex}

\noindent
\textbf{Theorem \ref{thm.security}}
\import{./}{proofs/security.tex}

\noindent
\textbf{Full proofs. }
\label{sec.proofs}
% \restateThmPredicatePersistence*
% \import{./}{proofs/predicatepersistence.tex}
\begin{lemma}[Local goodness]
\label{lem.localgood}
Assume $\chain$ contains only honestly-generated blocks in an optimistic
execution. For all levels $\mu$, for all constant $\delta > 0$, all continuous
subchains $\chain' = \chain[i:j]$ with $|\chain'| \geq m$ are locally good,
$\textsf{local-good}_{\delta}(\chain', \chain, \mu)$, with overwhelming
probability in $m$.
\end{lemma}
\import{./}{proofs/localgood.tex}

\begin{lemma}[Multilevel quality]\label{lem.multilevel}
For all $\mu, 0 < \delta \leq 0.5$, chain $\chain$ containing only
honestly-generated blocks in an optimistic execution has $(\delta, k_1)$
multilevel quality at level $\mu$ with overwhelming probability in $k_1$.
\end{lemma}
% \begin{proof}
% Identical.
% \Qed
% \end{proof}

\begin{lemma}[Superquality]
\label{lem.superquality}
For all $\mu, \delta > 0$, a chain $\chain$ adopted in an optimistic execution
has $(\delta, m)$-superquality at level $\mu$ with overwhelming probability in
$m$.
\end{lemma}
\import{./}{proofs/superquality.tex}

% \noindent
% \textbf{Theorem~\ref{thm.few-levels}}
\restateThmFewLevels*
\import{./}{proofs/fewlevels.tex}

% \noindent
% \textbf{Theorem~\ref{thm.large-expansion}}
\restateThmLargeExpansion*
\import{./}{proofs/largeexpansion.tex}

% \noindent
% \textbf{Lemma~\ref{lem.small-support}}
\restateThmSmallSupport*
\import{./}{proofs/smallsupport.tex}

% \noindent
% \textbf{Theorem~\ref{thm.succinctness}}
\restateThmSuccinctness*
\import{./}{proofs/succinctness.tex}

% \noindent
% \textbf{Theorem~\ref{thm.infix-security}}
\restateThmInfixSecurity*
\import{./}{proofs/infixsecurity.tex}

% \noindent
% \textbf{Theorem~\ref{thm.velvet-succinctness}}
\restateThmVelvetSuccinctness*
\import{./}{proofs/velvetsuccinctness.tex}
