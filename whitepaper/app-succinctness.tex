\section{More on succinctness}
\label{sec.app-succinctness}

\subsection{Proof of succinctness}
We now prove the optimistic succinctness claims of
Section~\ref{sec.succinctness}.

\restateThmFewLevels
\import{./}{proofs/fewlevels.tex}

\restateThmLargeExpansion
\import{./}{proofs/largeexpansion.tex}

\restateThmSmallSupport
\import{./}{proofs/smallsupport.tex}

\restateThmSuccinctness
\import{./}{proofs/succinctness.tex}

\subsection{Succinctness of adversarial proofs}
In the stronger adversarial setting, however, it is possible for the adversary
to produce large dummy (incorrect) proofs that expand the verification time;
security will not be hurt but it would take more time to complete verification.
One may dismiss this as a trivial denial of service attack and have a resource
bounded verifier simply stop if it is confronted with such a processing task.
However, simply dismissing superpolylogarithmic proofs is an incorrect strategy,
as honest provers can produce such longer proofs in case an adversarial miner
harms the goodness of the blockchain.

It would therefore be useful for honest provers to have the ability to
signal to the verifier that such time expansion is indeed necessary because of
an attack on superchain quality, rather than because a malicious prover is
simply sending long proofs that will eventually be rejected. With such signaling
mechanism, a resource bounded verifier can distinguish between a denial of
service attack that may be directed solely to it from  a denial of service
attack that  is launched by an attacker that has the ability to  interfere
globally with superchain quality.

To facilitate the above signaling, we offer a simple generalization of our
construction that achieves this. Our extended construction allows the verifier
to stop processing input early, in a streaming fashion, thereby only requiring
logarithmic communication complexity per proof received. To achieve this,
observe that honest proofs need to be large only if there is a violation of
\textit{goodness}. However, goodness is not harmed when the chain is not under
attack by the adversarial computational power or network. Therefore, we require
the prover to produce a \textit{certificate of badness} in case there is a
violation of \textit{goodness} in the blockchain. This certificate will always
be logarithmic in size and must be sent prior to the rest of the proof by the
prover to the verifier. Because the certificate will be logarithmic in size even
in the case of an adversarial attack on the chain, the honest verifier can stop
processing the certificate after a logarithmic time bound. If the certificate is
claimed to be longer, the honest verifier can reject early by deciding that the
prover is adversarial. Looking at the certificate, the honest verifier
determines whether there is a possibility for a lack of goodness in the
underlying chain. If there's no adversarial computational power in use, the
certificate is impossible to produce. We leave the full description of
certificates of badness for the full version of this paper.

The certificates of badness are produced easily as follows. First, the honest
verifier finds the maximum level $\text{max-}\mu$ at which there are at least
$m$ $\text{max-}\mu$-superblocks and includes it in the certificate. Then,
because there is a violation of goodness there must exist two levels $\mu <
\mu'$ such that $2^\mu|\chain\upchain^\mu| > (1 +
\delta)2^{\mu'}|\chain\upchain^{\mu'}|$ in some part $\chain$ of the honestly
adopted chain. But $\mu' - \mu \leq \text{max-}\mu$. Therefore, there must exist
two adjacent levels $\mu_1 < \mu_2$ which break goodness but with error
parameter $(1 + \delta)^{1/{\text{max-}\mu}}$. In particular, it will hold that
$2^{\mu_1}|\chain\upchain^{\mu_1}| > (1 +
\delta)^{1/{\text{max-}\mu}}2^{\mu_2}|\chain\upchain^{\mu_2}|$. This condition
is direct for the prover to find and trivial for the verifier to check and
completes the construction. Note that it is possible that a certificate of
badness is produceable where two adjacent levels have more than $(1 +
\delta)^{1/{\text{max-}\mu}}$ error even if there is no harm to global goodness;
however, these certificates cannot be produced when no adversarial power is in
use. The algorithm to do this is shown in Algorithm~\ref{alg.badness}.

\import{./}{algorithms/alg.badness.tex}

Therefore, we augment the NIPoPoW construction as follows. The honest prover
sends a tuple of two items. The first item is empty if the second item is
polylogarithmic in the size of the chain; otherwise it is a certificate of
badness. The second item is the NIPoPoW proof as in the previous construction.
The verifier processes only the first polylogarithmic number of bytes from the
incoming proof. If within that portion a certificate of badness is found, it is
checked for validity. If it is found to be valid, the whole proof is checked,
regardless of size. If it is found to be invalid or no certificate has been
provided, then the proof is rejected as invalid. We call the augmented
construction \textit{certified NIPoPoWs}.

\begin{restatable}[Certified NIPoPoWs succinctness]{lemma}{restateLemCertifiedSuccinctness}
    \label{lem.certified-succinctness}
    If all miners are honest and the network scheduling is random,
    certified non-interactive proofs-of-proof-of-work produced by the adversary
    are processed in polylogarithmic time in the size of the chain by honest
    verifiers, except with negligible probability in $m$.
\end{restatable}
\begin{proof}
    Because all miners are honest and the network scheduling is random,
    therefore certificates of badness exist with negligible probability in $m$.
    Conditioning on the event that certificates of badness do not exist, the
    honest verifier will reject the proof in polylogarithmic time.
    \Qed
\end{proof}

We also establish that the modified construction does not harm security below.
Security is established in the general case where the adversary has minority
mining power.

\begin{theorem}[Certified NIPoPoWs security]
    Assuming honest majority, certified non-interactive proofs-of-proof-of-work
    are secure, except with negligible probability in $\kappa$.
\end{theorem}
\begin{proof}
    We distinguish two cases: Either goodness has been violated; or it has not
    been violated. Suppose that goodness has been violated. In that case, an
    honest prover will include a certificate of badness in their proof and their
    proof will be processed by an honest verifier.

    In the case where goodness is not violated, all honest proofs will be
    logarithmic in size as established by
    Lemma~\ref{lem.certified-succinctness}. Therefore, all honest proofs will
    be processed by an honest verifier.

    Under the condition that all honest proofs will be processed, the rest of
    the security argument follows immediately from Theorem~\ref{thm.security}.
    \Qed
\end{proof}

We note that, compared to previous work~\cite{KLS}, the adversarial model in
which we have proven our succinctness is stronger in that the adversary is able
to produce proofs.

\subsection{Infix succinctness}
Having established the succinctness of the modified suffix construction, the
succinctness of the infix construction follows in the next corollary.

\begin{restatable}{corollary}{restateCrlyInfixSuccinctness}
\label{thm.infix-succinctness}
The infix NIPoPoW protocol $(P, V)$ is succinct for all computable
infix-sensitive $k$-stable predicates $Q$ in which the witness predicate $D$
depends on a \emph{constant} number of blocks $d$.
\end{restatable}
\import{./}{proofs/infixsuccinctness.tex}
