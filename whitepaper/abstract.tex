Blockchain protocols such as bitcoin provide decentralized consensus mechanisms
based on proof-of-work. In this work we introduce and instantiate a new
primitive for blockchain protocols called
Non-Interactive-Proofs-of-Proof-of-Work (NIPoPoWs) which can be adapted into
existing PoW-based cryptocurrencies. Unlike a traditional blockchain client
which must verify the entire linearly-growing chain of PoWs, clients based on
NIPoPoWs require resources only logarithmic in the length of the blockchain.
NIPoPoWs solve two important open questions for PoW based consensus protocols:
The problem of constructing efficient transaction verification (SPV) clients and
the problem of constructing efficient sidechain proofs. We provide a formal
model for NIPoPoWs. We prove our construction is secure and establish its
succinctness in a well defined model setting. We provide simulations and
experimental data to measure concrete communication efficiency and security. We
also present an attack against the only previously known (interactive) PoPoW
protocol that showcases the difficulty of designing such protocols. Finally, we
provide two ways that our NIPoPoWs can be adopted by existing blockchain
protocols, first via a soft fork, and second via a new update mechanism that we
term a ``velvet fork'' that enables harnessing some of the performance benefits
of NIPoPoWs even with a minority upgrade.
