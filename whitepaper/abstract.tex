Today's leading cryptocurrencies, Bitcoin and Ethereum, as well as myriad others, are built on a public consensus network based on proof-of-work (PoW) mining. In this work, we construct a new primitive called Non-Interactive-Proofs-of-Proof-of-Work (NIPoPoWs) that can be adapted into existing PoW-based cryptocurrencies to improve their performance and extend their functionality.
Unlike a traditional blockchain client
which must verify the entire linearly-growing chain of PoWs, clients based on
NIPoPoWs require resources only logarithmic in the length of the blockchain.

Our work improves and extends on prior work by Kiayias et al. Compare to theirs, our protocol is non-interactive, requiring only one round of communication. We also identify and correct a double-spend vulnerability and proof flaw in the prior work. %Correcting this required \anote{TODO}.

We provide empirical validation for NiPoPoWs through an implementation and benchmark study, in the context of two new applications:
First, we consider a multi-client blockchain that supports all proof-of-work currencies rather than just one, with up to 90\% reduction in bandwidth. %, \anote{XXX about efficiency}.
Second, we discuss a ``two-way-peg'' token that span multiple independent blockchains. To achieve this, we implemented the NiPoPoW verifier in the Solidity programming language, enabling one blockchain to act as a client for another.

 Finally, we present two deployment strategies by which our NIPoPoWs can be adopted by existing cryptocurrencies: first via a soft fork, and second via a new update mechanism that we
term a ``velvet fork'' that enables harnessing some of the performance benefits
of NIPoPoWs even if only a small number of miners adopt it.
